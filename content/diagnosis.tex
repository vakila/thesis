% DIAGNOSIS
%
% !TEX root = ../thesis-main.tex
%
\chapter{Diagnosis of lexical stress errors}
\label{chap:diagnosis}

%\cleanchapterquote{You can’t do better design with a computer, but you can speed up your work enormously.}{Wim Crouwel}{(Graphic designer and typographer)}


\section{Related work}

	Lots of Nancy references
	
	\cite{Duong2011}
	
	\cite{Probst2002}

\section{Automatic segmentation of nonnative speech}
	\subsection{Segmentation via forced alignment}
	\subsection{Evaluation of system accuracy}
	\subsection{Coping with segmentation errors}
		\cite{Mesbahi2011},
		\cite{Orosanu2012}
	
\section{Prosodic analysis}
	``acoustic differences between stressed and unstressed syllables are relatively large in spontaneous speech. With laboratory-read materials, however, such differences do not always arise'' \citep[p.~275]{Cutler2005}

	\subsection{Pitch}
		``...in German utterances, stressed syllables could be signaled by any F0 obtrusion from the overall contour, so that a stressed syllable could be either higher or lower in pitch than its neighbors'' \citep[p.~267]{Cutler2005}
	\subsection{Duration}
	\subsection{Intensity}
		least important of the three \citep{Cutler2005}
	
\section{Comparison of native and nonnative speech}
		\cite{Probst2002}
	\subsection{Using a single reference speaker}
		\subsubsection{Manually selecting a reference}
		\subsubsection{Automatically selecting a reference}
	\subsection{Using multiple reference speakers}
	\subsection{Using no reference speaker?}
		\cite{Duong2011}

		
\section{Summary}