% !TEX root = Clean-Thesis.tex
%
%\pdfbookmark[0]{Résumé}{Résumé}
\chapter*{Résumé}
\label{sec:abstract:fr}
\vspace*{-10mm}

La réalisation prosodique de l'accent lexical, c.-à-d. l'accentuation plus ou moins forte d'une ou de plusieurs syllabes d'un mot, est un élément important de la phonologie allemande qui peut s'avérer très problématique pour les personnes qui apprennent l'allemand en tant que langue étrangère (L2). Les personnes de langue maternelle (L1) française sont particulièrement mal armées pour l'appréhender puisque la réalisation de l'accent est différente (voire inexistante) en prosodie lexicale française. Les exercices de prononciation demandent souvent une attention individuelle considérable de la part de l'enseignant qui n'est pas toujours envisageable dans le contexte des cours de langue en classe. L'Enseignement de la prononciation assisté par ordinateur (EPAO) apparaît depuis quelques décennies comme une utilisation prometteuse de la technologie pour l'enseignement individualisé de la prononciation en l'absence d'un enseignant de chair et d'os. Ce mémoire s'intéresse aux possibilités qu'offre l'EPAO pour aider les francophones qui apprennent l'allemand comme langue étrangère à mieux maîtriser l'accent lexical.

Afin de déterminer la nature des erreurs d'accent lexical des francophones qui apprennent l'allemand, ce mémoire décrit comment ces erreurs ont été recensées manuellement dans un corpus vocal et analyse la fréquence et les types d'erreurs observées. Il explore ensuite différentes méthodes pour diagnostiquer automatiquement ces erreurs, dont certaines méthodes inédites qui opèrent une classification par le biais de l'apprentissage sur ordinateur supervisé ou qui comparent les prononciations des apprenants avec un ou plusieurs énoncés de référence, prononcé(s) par une personne de langue maternelle allemande. Ce mémoire s'intéresse ensuite aux utilisations possibles de ces diagnostiques pour informer les étudiants de leurs erreurs, parfois d'une façon encore inédite en EPAO appliqué à l'allemand.

L'élément principal de ce mémoire reste cependant sa description d'un prototype d'outil d'EPAO, \TODO{de-stress}, qui combine ces différents diagnostiques et méthodes de retour d'information, présentés à l'aide d'une interface web facile d'utilisation. Son intérêt ne réside pas seulement dans la communication aux apprenants  de leurs erreurs d'accentuation lexicale, mais aussi dans le fait qu'il facilite la recherche sur l'efficacité des différents types de diagnostiques et de retours examinés dans ce mémoire. Il permet également aux professeurs d'allemand comme langue étrangère de créer des exercices adaptés aux besoins de leurs élèves. Ainsi, cet outil constitue une étape importante dans le développement d'un système EPAO complet et intelligent pour les francophones qui apprennent l'allemand.

\begin{flushright}
\textit{Traduit de l'anglais par Marie Springinsfeld}
\end{flushright}