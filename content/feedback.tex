% FEEDBACK
%
% !TEX root = ../thesis-main.tex
%
\chapter{Feedback on lexical stress errors}
\label{chap:feedback}

%\cleanchapterquote{You can’t do better design with a computer, but you can speed up your work enormously.}{Wim Crouwel}{(Graphic designer and typographer)}

Since the focus of this thesis is on pronunciation training, not pronunciation assessment (see \cref{sec:capt:systems}), feedback on the errors diagnosed via the methods described in \cref{chap:diagnosis} is an important component of the prototype CAPT tool developed in this work. As mentioned in \cref{sec:capt:l2ed}, the particular importance of corrective feedback in pronunciation training is generally acknowledged,
%\citep{Neri2002,Dlaska2013}, 
though much remains to be learned about when and how feedback can be most effective. Therefore, an important contribution of this thesis is the creation of a feedback module for the lexical stress CAPT tool which offers a variety of possible feedback types, and a Graphical User Interface (GUI) allowing a researcher or instructor to easily switch between feedback types. The hope is that researchers can use this modular tool in in-vivo studies with language learners to compare the effects of various feedback types on the acquisition of L2 German prosody by L1 French speakers (or perhaps even speakers of other L1s); though it is outside the scope of the thesis to carry out 
such studies,
%in vivo studies with learners to determine which feedback types are most effective in which situations, 
the tool has been designed 
with this application in mind.
%to facilitate such studies going forward. 
Ultimately, once research has given us a better understanding of which feedback types are most effective in which situations, the modular feedback delivery system developed here could theoretically be embedded in a full-featured intelligent tutoring system, where models of the relevant learning contexts (such as  the objectives of the current exercise, or the student's past achievements and personal goals and preferences) could be used to automatically select the most useful feedback type to present to the learner, as mentioned in \cref{sec:intro:objectives} and illustrated in \cref{fig:hourglass-ITS,fig:feedback}.

This chapter presents the various options for the types of feedback that can be generated given a diagnosis of the learner's lexical stress realization, 
\TODO{\textit{rephrase:} guided by the notion that} to maximize its utility in future feedback research, the CAPT tool should offer as wide a variety of feedback options as possible, especially those offering types of feedback not commonly seen in existing CAPT systems. 
	Some of these feedback options are visual \cref{sec:fb:visual}, some auditory \cref{sec:fb:auditory}, \TODO{\textit{reword:} and others constitute alternative feedback types} \cref{sec:fb:alternative}, as illustrated in \cref{fig:feedback}. 

%\textcite{Hattie2007} suggest that feedback in education should help the learner recognize their current learning goals (``Where am I going?''), assess their progress towards them (``How am I going?''), and identify next steps for achieving and moving beyond these goals (``Where to next?'').


%Feedback is important \citep{Neri2002}

%Since our focus is on pronunciation training and not just pronunciation assessment.



	\begin{figure}[htb]
		\includegraphics[width=\textwidth]{img/feedback}
		\caption{Delivery of prosody feedback in different modalities. \TODO{redo or remove}}
		\label{fig:feedback}
	\end{figure}

%\section{Related work}
%\label{sec:fb:related}
%
%	\cite{Sitaram2011}
%	\cite{Bonneau2011}
%	 	\citep{Hattie2007}


\TODO{Restructure? as:}


	\section{Implicit feedback types}
	
		\TODO{Describe what implicit fb is}
	
		\subsection{Visual }
		
		Visual delivery of feedback on learner errors (or lack thereof) is a widely used technique in CAPT.
%	\subsection{Visualizations of the speech signal}
%	\label{sec:visual:visualizations}
	In many existing CAPT tools %such as Snoori \citep{Bonneau2011,Parole2013} and WinPitch LTL \citep{Martin2004}, 
	(e.g. \cite{Martin2004,Henry2007}),
the learner is presented with relatively direct visualizations of the speech signal, such as its waveform (oscillogram) and spectrogram, often with overlays highlighting perceptually relevant properties such as the pitch contour and durations of various parts of the utterance. 
	Indeed, this is the case in Jsnoori, as seen in \TODO{Jsnoori screenshot}.
	However, as \textcite{Neri2002} point out, waveforms and spectrograms are signal representations designed for speech researchers, not language learners, and the latter may have difficulty understanding these visualizations without the proper training. To research whether this conjecture holds, these direct visualizations must be compared with alternatives in user studies with learners; to this end, visual feedback in the lexical stress CAPT tool developed in this thesis project focuses on alternatives to direct signal visualizations, as described in this section.	
		
			\subsubsection{Graphical abstractions of prosody}
			\label{sec:implicit:visual:graphical}
			
			One type of alternative to direct visualizations of the speech signal is a more abstract graphical representation of the lexical stress pattern in the native reference speaker and/or the learner's speech. Classroom materials for pronunciation instruction sometimes represent lexical stress patterns using dots or other shapes, one for each syllable, whose relative sizes indicate each syllable's prominence in the word \citep{Hirschfeld1998}. By mapping the acoustic features of each syllable in the utterance(s) to graphical features (e.g. height, width) of a geometrical shape
			%-- e.g. using width to represent duration, height to represent F0, \TODO{and opacity to represent intensity} -- 
			it is possible to dynamically create a visual abstraction of the relevant properties of the learner's utterance as well as those of the reference utterance(s). This abstracted visual representation of prosody may be easier for the learner to interpret than the complex and possibly overwhelming signal visualizations more commonly used to give prosodic feedback in CAPT as mentioned in the previous section.
			
			\Cref{fig:rectangles} illustrates the display of such graphical abstractions in the prototype CAPT tool. Each syllable in an utterance is represented by a rectangle, the length of which corresponds to the duration of that syllable (as a percentage of the total word duration), the height of which represents the mean F0 in that syllable (normalized by dividing the absolute mean F0 for the syllable by the overall mean in the word), and the opacity of which corresponds to the mean intensity of that syllable (again normalized by dividing by the mean in the entire word). If the learner hovers their mouse over one of the rectangles, they are presented with the exact values for each of these features in a tooltip overlay, which can be seen as the small yellow box in the central area of \cref{fig:rectangles}.
			
				\begin{figure}
					\centering
					\caption[Feedback via graphical abstractions of prosody]{Screenshot of feedback via graphical abstractions of prosody}
					\includegraphics[width=\textwidth]{img/screenshots/rectanglesWithOpacity}
					\label{fig:rectangles}
				\end{figure}			
			
			In the prototype CAPT tool, the mappings between graphical properties (width, height, and opacity of the rectangle) and prosodic features (duration, F0, and intensity, respectively) are hard-wired. However, researchers using the system to experiment with different feedback methods should ideally be able to change the mapping or omit one of the features if necessary, so a more flexible feature-mapping mechanism would be a worthwhile improvement to the system (see \cref{sec:conclusion:future}).
			
			
			\subsubsection{Stylized text}
			\label{sec:implicit:visual:text}
			
			A related approach to the abstract graphical representations just described involves stylizing, or reshaping, the text of the word(s) pronounced  to match the prosodic features of the learner's utterance.
	This is essentially the approach used by the work \textcite{Sitaram2011}, who used text stylization to help visualize prosody in the Project LISTEN reading tutor (see \cref{sec:capt:listen}. 
	Such text stylization is also often used to convey canonical prosody in pronunciation instruction materials \citep{Behme-Gissel2005,Hirschfeld2007a}, e.g. by using larger text for the  stressed syllable than for the unstressed syllable(s) in a given word.
	Familiarity with this type of presentation might make feedback via stylized text easier for learners to comprehend, so text stylization was another form of implicit visual feedback implemented in the system, as illustrated in \cref{fig:textstylization}.
			
			
			\begin{figure}
			\centering
			\caption[Feedback on syllable duration via text stylization]{Screenshot of feedback on syllable duration via text stylization}
			\includegraphics[width=\textwidth]{img/screenshots/textStylization}
			\label{fig:textstylization}
			\end{figure}
		
		
		When using geometric shapes to visualize prosody, different prosodic features can be conveyed simultaneously by mapping each to a different geometric property, as described in \cref{sec:implicit:visual:graphical} However, when dealing with text, visualizing multiple prosodic features at the same time is more difficult.
		First of all, noticing a clear difference between two syllables in terms of textual features such as height, font weight, or letter spacing is not as easy as comparing the height and width of two rectangles, given the inherent geometric variability of the different letters of the alphabet.
		Secondly, if text is stretched or skewed too dramatically, it becomes more difficult to read, which may be distracting for learners using the system. 
		
		
		Therefore, in the CAPT tool developed here, the text of a given syllable is stylized with a simple mapping between font size (as a multiple of the default size) and duration (as a fraction of the word duration). Duration was chosen as the prosodic feature to visualize based on its relative importance for the perception of lexical stress in German (see \cref{chap:diagnosis}). Font size was chosen as the textual feature to manipulate because it can be changed without distorting the text, i.e. without risking decreased legibility. Of course, other mappings between prosodic and textual features could be imagined, and once again the addition of other options than those currently implemented in the system could be worthwhile (see \cref{sec:conclusion:future}), though this might be less useful for text stylization than for  graphical visualizations, for the reasons mentioned in the previous paragraph.
		
			
			
		\subsection{Auditory}
			\subsubsection{Student \& reference audio}
			\subsubsection{Resynthesized audio}
			
	\section{Explicit feedback types}
		\subsection{Skill bars}
		\subsection{correct/incorrect/none classification}
		\subsection{Verbal assessment (Jsnoori messages)}
		\subsection{Letter grades?}
		
	\section{Other types of feedback}
		\subsection{Self-assessment}
		\subsection{Metalinguistic feedback}


\section{\TODO{REMOVE OLD SECTIONS BELOW}}

\section{Visual feedback}
\label{sec:fb:visual}

%	Visual delivery of feedback on learner errors (or lack thereof) is a widely used technique in CAPT.
%%	\subsection{Visualizations of the speech signal}
%%	\label{sec:visual:visualizations}
%	In many existing CAPT tools %such as Snoori \citep{Bonneau2011,Parole2013} and WinPitch LTL \citep{Martin2004}, 
%	(e.g. \cite{Martin2004,Henry2007}),
%the learner is presented with relatively direct visualizations of the speech signal, such as its waveform (oscillogram) and spectrogram, often with overlays highlighting perceptually relevant properties such as the pitch contour and durations of various parts of the utterance. 
%	Indeed, this is the case in Jsnoori, as seen in \TODO{\cref{Jsnoori screenshot}}.
%	However, as \textcite{Neri2002} point out, waveforms and spectrograms are signal representations designed for speech researchers, not language learners, and the latter may have difficulty understanding these visualizations without the proper training. To research whether this conjecture holds, these direct visualizations must be compared with alternatives in user studies with learners; to this end, visual feedback in the lexical stress CAPT tool developed in this thesis project focuses on alternatives to direct signal visualizations, 
%	several types of which are presented in this section.
	
	
%	\subsection{Graphical representations of prosody}
%	\label{sec:visual:graphical}
%	


%	\TODO{\textit{DE-PROPOSAL-IZE} One type of alternative would be a more abstract graphical representation of the lexical stress pattern in the native reference speaker and/or the learner's speech. Classroom materials for pronunciation instruction sometimes represent lexical stress patterns using dots or other shapes, one for each syllable, whose relative sizes indicate each syllable's prominence in the word \citep{Hirschfeld1998}. This type of visualization would be relatively simple to implement, given that the reference or learner utterance can be classified into one of a set of stress patterns \citep{Kim2011,Shahin2012a}. It would also be possible to map the acoustic features of each syllable in the utterance(s) to graphical features of the representative shape, e.g. using size to represent duration, vertical position to represent F0, and darkness to represent intensity. 
	%MOVED TO FUTURE WORK
	%To facilitate studies on which mappings, if any, make this feedback useful to the learner, the researcher-facing GUI should offer control over the different possible mappings.}
	
%	\begin{figure}
%		\centering
%		\caption{Screenshot of feedback on syllable duration via graphical shapes \TODO{explain tooltip}}
%		\includegraphics[width=\textwidth]{img/screenshots/rectangles}
%		\label{fig:visual:graphical}
%	\end{figure}
	
	
%	\subsection{Stylized text}
%	\label{sec:visual:text}
%	
%	In a related approach used by \textcite{Sitaram2011}, instead of displaying an abstract graphical representation of 	features of the learner's utterance, the text of the word(s) pronounced is stylized, or reshaped, to match these features.
%	%This is essentially the approach used by \textcite{Sitaram2011}, though they modify the text of each word instead of a more abstract visual representation. 
%	Such text stylization is also often used to convey canonical prosody in pronunciation instruction materials \citep{Behme-Gissel2005,Hirschfeld2007a}, 
%	and familiarity with this type of presentation might make feedback via stylized text easier for learners to comprehend; therefore,
%	including stylized text as a feedback option in the CAPT tool is a logical choice.
%	%so it would be logical for the CAPT tool to offer text stylization as a feedback option.
%	
%	\TODO{write up text stylization}	
%	
%	\begin{figure}
%	\centering
%	\caption[Feedback on syllable duration via text stylization]{Screenshot of feedback on syllable duration via text stylization  \TODO{crop out banner}}
%	\includegraphics[width=\textwidth]{img/screenshots/textStylization}
%	\label{fig:visual:text}
%	\end{figure}
	
	
	%MOVED TO FUTURE WORK 
	%As with the shapes mentioned above, and following \textcite{Sitaram2011}, it would be interesting to explore the possible mappings between acoustic features and properties of the text of each syllable (e.g. size, weight, underlining/decoration, etc.), with these mappings controllable by the researcher via the GUI.

%Underlining \citep{Hirschfeld2007a}

%Bold + relative position of syllables \citep{Behme-Gissel2005}

\subsection{Other}
\label{sec:visual:other}


	\TODO{Given some visual representation of the learner's utterance, be it textual or more abstract, visual feedback should also be given on what the learner can do to improve their lexical stress realization. \textcite{Bonneau2011} deliver such feedback in the F0 dimension by displaying arrows which indicate whether the user should raise or lower the pitch of a given syllable to make their realization more like that of the reference speaker, and this is one option for the CAPT tool. Another might be the use of animation to transform the visualization of the learner's (incorrectly realized) utterance into a corresponding visualization of the correct realization, e.g. by growing or shrinking the size of the dot or text for each syllable to visualize the desired change in duration, or showing it moving up or down to convey the desired change in pitch.}
	
	
%	Visualizations of the required articulators, such as those displayed in the Fluency pronunciation trainer \citep{Eskenazi2000}, may be helpful for correcting certain segmental errors, but are not likely of much use for correcting lexical stress, and will therefore not be implemented in the proposed tool.

\TODO{	Implementation of at least one visual feedback type will be of high priority in this work. Stylized text and graphical representations will be explored first. If time allows, animation will be added to convey corrective feedback to the learner.
}	
	
\section{Auditory feedback}
\label{sec:fb:auditory}

%Some have speculated that auditory feedback may be more helpful than visual feedback in making learners more aware of lexical stress patterns \citep[p.~3]{Bissiri2009}, so it will be important to address both in the proposed CAPT tool.

In foreign language classrooms, feedback on correct pronunciation is often given implicitly by
%One strategy often used in classrooms is 
%%implicit feedback, i.e. 
allowing the learner to listen to a native speaker's production of the target utterance and/or a recording of their own production.
% while there is no reason not to include the option for this type of implicit feedback in the CAPT tool, it will also be necessary to deliver explicit auditory feedback, as the latter may improve pronunciation more \citep{Dlaska2013} 
	However, 
%as described in \cref{sec:capt:systems} above, 
previous work on delivering lexical stress feedback 
(see \cref{sec:capt:systems}) 
has revealed that learners seem to benefit more from prosodically modified implicit feedback, either in the form of a learner utterance  modified to reflect the ``correct'' prosody of a native reference utterance \citep{Bonneau2011}, or a native utterance modified to place exaggerated emphasis on the stressed syllable \citep{Bissiri2006,Bissiri2009}.
%	%In previous work on delivering %explicit 
%%feedback on lexical stress, as described in \cref{sec:capt:systems}, resynthesis of the learner's utterance has often been performed, where prosodic modifications are applied to make the utterance match the ``correct'' prosody of a reference utterance \citep{Bonneau2011,Bissiri2006,Bissiri2009}. 
%%	Another possibility to be explored follows \citeauthor{Bissiri2009} (\citeyear{Bissiri2006,Bissiri2009}), who found that presenting learners with native model utterances which had been modified to place exaggerated emphasis on the stressed syllable gave positive results. 
%	Therefore, these two types of modification may be implemented in the proposed CAPT tool's feedback module; the prosodic modification capabilities of the Jsnoori program \citep{Parole2013} will be used to transform the given utterance. 
%	If a more generalized model of the native prosody of a given word is developed (see \cref{sec:diag:compare}), it might also be possible to modify a reference (or learner) utterance to match this speaker-independent prosody.
	
	At least one type of audio feedback type will be implemented in the CAPT tool, with the highest-priority option being prosodic modification of the learner's utterance to match a single, manually-selected reference utterance, following \textcite{Bonneau2011}; Jsnoori \citep{Parole2013} will be used to perform this modification. If a generalized lexical stress model is successfully integrated into the diagnostic module (see \cref{sec:diag:compare}), the next highest-priority task will be performing prosodic modification of the learner's utterance based on this model. Emphasizing stressed syllables in the native reference utterance(s) will be of lowest priority. 
	
	\TODO{Resynthesis in Jsnoori
	\begin{itemize}
		\item TD-PSOLA
		\item Pitch marking
		\item ...?
	\end{itemize}
	}
 


%\citep{Jilka1998}?

%TODO replace
%	\subsection{Enhanced reference utterance}
%	\label{sec:auditory:enhanced}
%	
%TODO replace
%	\subsection{Resynthesized learner speech}
%	\label{sec:auditory:resynth}
	
	
	
\section{Alternative feedback types}
\label{sec:fb:alternative}

Other options, which will only be explored if time allows, include (in order of priority) feedback encouraging self-assessment and self-correction, metalinguistic feedback, and interactivity. %TODO explain self-regulated learning
 	Self-assessment and self-correction can be encouraged by presenting learners with targeted questionnaires before delivering diagnosis and feedback, e.g. asking learners to listen to their utterance and assess whether they have placed stress on the correct syllable, or 
 	%asking them to listen to an incorrect production (theirs or another speaker's) and 
 	asking how the speaker of an incorrect production could have realized stress properly (``By making the first syllable longer'', etc.). 
	Metalinguistic feedback, e.g. reminding learners %who have made an error 
of the stress rule(s) affecting the target utterance, could be delivered either visually (e.g. text displayed on the screen), auditorily (e.g. playback of an instructor's voice), or both. 
 	Interactivity could be achieved by allowing learners to interact with the resynthesis component to modify the prosody of their utterance, as is done in WinPitch LTL \citep{Martin2004}. %TODO {Snoori too?} 
 	By allowing researchers to easily control which of these feedback options to present to the learner, the tool could facilitate research into the effects of alternative feedback types such as these, which have not yet been adequately studied in CAPT.
 	
	\subsection{Self-assessment}
	\label{sec:alternative:selfassess}
	
	\TODO{}
 	
 	
 	\begin{figure}
		\centering
		\caption{Self-assessment questionnaire presented to learner before feedback delivery}
		\includegraphics[width=.7\textwidth]{img/screenshots/selfAssessment}
		\label{fig:visual:selfassess}
	\end{figure}
 

%TODO replace
%	\subsection{Metalinguistic feedback}
%	\label{sec:alternative:metaling}
%	
%TODO replace
%	\subsection{Interactive feedback}
%	\label{sec:alternative:interactive}
%	
%TODO replace
%	\subsection{Implicit feedback}
%	\label{sec:alternative:implicit}


\section{\TODO{Controlling feedback methods in the system}}
\label{sec:fb:system}
\TODO{}

\section{Summary}
\label{sec:fb:summary}
\TODO{}