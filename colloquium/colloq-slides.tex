\documentclass[xcolor={dvipsnames}]{beamer}

%%% PACKAGES %%%
\usepackage{graphicx}
\usepackage{tabularx}
\usepackage{booktabs}
\usepackage{multirow}
%\usepackage[usenames,dvipsnames]{color}
\usepackage{textpos}
\usepackage{tipa}
\usepackage{hyperref}
%\usepackage{pifont}

%\usepackage{caption}[2008/08/24]
%\usepackage{subcaption}

%\usepackage{enumitem}

\usepackage{perpage}
\MakePerPage{footnote}

%%% TODOs %%%
\usepackage[dvipsnames]{xcolor}
\newcommand{\TODO}[1]{{\color{red}\textbf{[TODO #1]}}}

%%% FONT %%%
\usepackage{tgheros}

%%% BEAMER THEME %%%
\usetheme{default}
\usecolortheme{whale}
\usecolortheme{orchid}
\setbeamertemplate{navigation symbols}{} 
\setbeamertemplate{footline}[frame number]
%\setbeamertemplate{footline}{%
%  \leavevmode%
%  \hbox{%
%    \pgfsetfillopacity{0}\begin{beamercolorbox}[wd=.333333\paperwidth,ht=2.25ex,dp=1ex,center]{author in head/foot}%
%      \usebeamerfont{author in head/foot}\pgfsetfillopacity{1}\insertshortauthor
%    \end{beamercolorbox}%
%    \pgfsetfillopacity{0}\begin{beamercolorbox}[wd=.333333\paperwidth,ht=2.25ex,dp=1ex,center]{title in head/foot}%
%      \usebeamerfont{title in head/foot}\pgfsetfillopacity{1}\insertshorttitle
%    \end{beamercolorbox}%
%    \pgfsetfillopacity{0}\begin{beamercolorbox}[wd=.333333\paperwidth,ht=2.25ex,dp=1ex,right]{date in head/foot}%
%      \usebeamerfont{date in head/foot}\pgfsetfillopacity{1}\insertshortdate{}\hspace*{2em}
%      \insertframenumber{} / \inserttotalframenumber\hspace*{2ex}
%    \end{beamercolorbox}}%
%  \vskip0pt%
%}
\addtobeamertemplate{frametitle}{}{%
\begin{textblock*}{100mm}(.875\textwidth,-.9cm)
\includegraphics[height=.9cm]{uds-logo-text-white.png}
\end{textblock*}}
\setbeamertemplate{caption}{\insertcaption}
\setbeamerfont{caption}{size=\scriptsize}
\setbeamertemplate{itemize subitem}[circle]
%\setbeamertemplate{bibliography item}[text]
\setbeamertemplate{bibliography item}[triangle]
\setbeamertemplate{blocks}[rounded][shadow=true]


%%% COLORS %%%
\definecolor{MutedBlue}{RGB}{83,121,170}
\definecolor{UniBlue}{RGB}{0,71,116}
\definecolor{DarkPink}{RGB}{178,56,119}
\definecolor{LightPink}{RGB}{255,224,246}
\definecolor{LightBlue}{RGB}{214,255,255}

%\setbeamercolor{title}{fg=UniBlue}
%\setbeamercolor{frametitle}{fg=UniBlue}
\setbeamercolor{author in head/foot}{fg=UniBlue}
\setbeamercolor{title in head/foot}{fg=UniBlue}
\setbeamercolor{date in head/foot}{fg=UniBlue}
\setbeamercolor{page number in head/foot}{fg=UniBlue}
\setbeamercolor{normal text}{fg=darkgray}
\setbeamercolor{structure}{fg=UniBlue}
\setbeamercolor{itemize item}{fg=UniBlue}
\setbeamercolor{itemize subitem}{fg=UniBlue}
%\setbeamercolor{section in toc}{fg=MutedBlue}
%\setbeamercolor{subsection in toc}{fg=darkgray}
%\setbeamercolor{bibliography entry author}{fg=darkgray}
%\setbeamercolor{bibliography entry title}{fg=darkgray}
%\setbeamercolor{bibliography entry location}{fg=gray!80!black}
%\setbeamercolor{bibliography entry note}{fg=gray!90!black}
%\setbeamercolor{bibliography entry note}{fg=darkgray}
\setbeamercolor{block title}{bg=structure,fg=white}
%\setbeamercolor{block body}{bg={palette secondary}}


%%% BIBLIOGRAPHY %%%
\usepackage[%
	backend=bibtex,
	citestyle=authoryear,
	maxcitenames=2,
	maxbibnames=99,
	firstinits=true,
	url=false,
	doi=false,
	isbn=false, 
	%sorting=none,
	]{biblatex}
\addbibresource{../library.bib}
% to cite on the same slide use \footfullcite{jones00}
\renewcommand{\footnotesize}{\scriptsize}

%% TITLE PAGE INFO %%
\author[A. Vakil]{Anjana Sofia Vakil
%\\\texttt{[anjanav,apalmer]@coli.uni-saarland.de}
} 
\institute{\includegraphics[height=4.5em]{uds-logo-text.png}\\Department of Computational Linguistics and Phonetics\\University of Saarland, Saarbr{\"u}cken, Germany}
\title{Automatic diagnosis and feedback for lexical~stress errors in non-native speech:\\
Towards a CAPT system for\\French learners of German}
%\subtitle{Investigating the impact of source language choice} 
\date[16.4.15]{Master's Thesis Colloquium\\16 April 2015}

%% DOCUMENT %%
\begin{document}
{
\setbeamertemplate{footline}{} 
\begin{frame}
  \titlepage
\end{frame}
}
\addtocounter{framenumber}{-1}





\begin{frame}{Lexical stress}
Some syllable(s) in a word more accentuated/prominent\footfullcite{Cutler2005}

\begin{center}
\begin{tabular}{ccc}
um$\cdot$FAHR$\cdot$en & vs. & UM$\cdot$fahr$\cdot$en \\
\textit{to run over} & & \textit{to drive around}\\
\end{tabular}
\end{center}

%\vspace{-1em}

\addtocounter{footnote}{-1}
\begin{itemize}
\item{German: variable stress placement, contrastive stress\footnotemark
%\footfullcite{Cutler2005}
}
\item{French: no word-level stress, final syllable lengthening\footfullcite{Michaux2013}}
\end{itemize}

\vfill

Goal: Computer-Assisted Pronunciation Training (CAPT) for lexical stress errors for French learners of German


\end{frame}

\begin{frame}
\frametitle{Outline}
\tableofcontents%[pausesections]
\end{frame}




\section{Motivation}
\begin{frame}{Motivation}
		\begin{figure}
			\centering
			\includegraphics[width=.8\textwidth]{../img/error-venn}
			\caption{Criteria for selecting errors to target in a CAPT system.}
			\label{fig:errors}
		\end{figure}
\end{frame}

\begin{frame}{Motivation}
Lexical stress errors seem to be: 
\begin{itemize}
\item Frequently produced by French learners of variable-stress languages\footfullcite{Bonneau2011}$^{,}$\footfullcite{Michaux2012}
\item More important for intelligibility in L2 German than other types of errors\footfullcite{Hirschfeld1994}
\item Possible to identify automatically by comparison$^1$ or classification\footfullcite{Kim2011}
\end{itemize}

\end{frame}

%\begin{frame}{Motivation}
%Frequently produced:
%\begin{itemize}
%\item French speakers perceptually ``deaf'' to Spanish stress\footfullcite{Dupoux1997}
%\item French learners make stress errors in Dutch\footfullcite{Michaux2012} and English\footfullcite{Bonneau2011}
%%\item French learners of Dutch (incorrectly) stress final syllables\footfullcite{Michaux2012}
%%\item French beginners frequently misplace English stress\footfullcite{Bonneau2011}
%\end{itemize}
%
%Affecting intelligibility:
%\begin{itemize}
%\item Prosodic errors impede intelligibility more than segmental errors
%\item Lexical stress errors have particularly strong impact on L2 German intelligibility\footfullcite{Hirschfeld1994}
%\end{itemize}
%
%
%Automatically detectable:
%
%\end{frame}

\section{Lexical stress errors by French learners of German}
		\begin{frame}{Lexical stress errors in learner speech}
		\begin{itemize}
		\item{How reliably can human annotators identify errors in learner utterances?}
		\vspace{1em}
		\item{How frequently are  errors actually produced by French learners of German?}
		\end{itemize}
		

		\end{frame}
		
	\subsection{Annotation of a learner speech corpus}
		\begin{frame}{Error annotation}
		Data: IFCASL corpus of French-German L1/L2 speech\footfullcite{Fauth2014}
		\begin{itemize}
			\item{German utterances by French and German speakers}
			\item{Word- and phone-level segmentations\\(syllable level added automatically)}
			\item{Selected 12 word types (bisyllabic, initial stress)}
			\item{Dataset: 668 word utterances by 55-56 speakers}
			%\item{No lexical stress error annotation}

		\end{itemize}

		\vfill		
		
		Annotators (15 in total):
		\begin{itemize}
			\item{L1: 12 German speakers, 2 English, 1 Hebrew}
			\item{Expertise: 2 Experts, 10 Intermediates, 3 Novices}
			\item{Annotated 3 word types in 1 $\sim$15 min. session}
		\end{itemize}
		
		\end{frame}
		
		\begin{frame}{Error annotation}
		\begin{figure}
			\centering
			\includegraphics<1>[height=.75\textheight]{../img/screenshots/AnnotationTool}
			\includegraphics<2>[height=.75\textheight]{AnnotationTool-withLabels}
			\caption{Praat annotation tool}
			%\caption[A screenshot of the graphical annotation tool scripted in Praat.]{A screenshot of the graphical annotation tool scripted in Praat. Green buttons allow the annotator to listen to  the word and sentence utterances. Gray buttons allow the annotator to record their judgment of stress accuracy; from top to bottom, the buttons correspond to the labels [correct], [incorrect], [none], [bad\_nsylls], and [bad\_audio].}
			\label{fig:annotationtool}
		\end{figure}
		\end{frame}
		
%		\begin{frame}{Error annotation}
%		\begin{itemize}
%			\item{
%			12 word types extracted from IFCASL sentences
%			\begin{itemize}
%				\item{bisyllabic (simple syllable comparison)}
%				\item{initial stress (difficult for French speakers)}
%			\end{itemize}
%			}
%			\item{668 word utterances in total (55-56 speakers)}
%			\item{
%			Multiple annotators label each word utterance as: 
%						\vspace{.25em}
%			{\small
%
%			\begin{tabular}{rll}
%				& $[$correct$]$ & stress on initial syllable \\
%				& $[$incorrect$]$ & stress on final syllable \\
%				& $[$none$]$ & no clear stress \\
%				& $[$bad\_nsylls$]$ & incorrect number of syllables \\
%				& $[$bad\_audio$]$ & technical problems \\
%			\end{tabular}
%			}
%%			\begin{tabular}{rll}
%%				{\footnotesize\color{UniBlue} $\bullet$} & $[$correct$]$ & stress on initial syllable \\
%%				$\bullet$ & $[$incorrect$]$ & stress on final syllable \\
%%				$\bullet$ & $[$none$]$ & no clear stress \\
%%				$\bullet$ & $[$bad\_nsylls$]$ & incorrect number of syllables \\
%%				$\bullet$ & $[$bad\_audio$]$ & technical problems \\
%%			\end{tabular}
%%%			\begin{itemize}
%%%				\item{[correct]: stress on initial syllable}
%%%				\item{[incorrect]: stress on final syllable}
%%%				\item{[none]: no clear stress}
%%%				\item{[bad\_nsylls]: incorrect number of syllables}
%%%				\item{[bad\_audio]: technical problems}
%%%			\end{itemize}
%			}
%		\end{itemize}
%		
%		\end{frame}
%		
		
	\subsection{Inter-annotator agreement}
		\begin{frame}{\TODO{}}
		\TODO{}
		\end{frame}	
	\subsection{Frequency of errors}
		\begin{frame}{\TODO{}}
		\TODO{}
		\end{frame}

\section{Error diagnosis}
	\subsection{Word prosody analysis}
		\begin{frame}{\TODO{}}
		\TODO{}
		\end{frame}		
	\subsection{Diagnosis by comparison}
		\begin{frame}{\TODO{}}
		\TODO{}
		\end{frame}
	\subsection{Diagnosis by classification}
		\begin{frame}{\TODO{}}
		\TODO{}
		\end{frame}

\section{Feedback}
	\subsection{Implicit}
		\begin{frame}{\TODO{}}
		\TODO{}
		\end{frame}
	\subsection{Explicit}
		\begin{frame}{\TODO{}}
		\TODO{}
		\end{frame}
	\subsection{Self-assessment}
		\begin{frame}{\TODO{}}
		\TODO{}
		\end{frame}

	
\section{The de-stress CAPT tool }
%	\subsection{For students}
%		\begin{frame}{\TODO{}}
%		\TODO{}
%		\end{frame}
%	\subsection{For teachers/researchers}
		\begin{frame}{\TODO{}}
		\TODO{}
		\end{frame}


\end{document}