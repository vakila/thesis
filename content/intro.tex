% INTRODUCTION
%
% !TEX root = ../thesis-main.tex
%
%\part{Introduction}
\chapter{Introduction}
\label{chap:intro}

%\cleanchapterquote{You can’t do better design with a computer, but you can speed up your work enormously.}{Wim Crouwel}{(Graphic designer and typographer)}




\section{Context: The IFCASL project}
\label{sec:intro:ifcasl}

The work reported here has been conducted in the context of the ongoing research project ``Individualized Feedback in Computer-Assisted Spoken Language Learning (IFCASL)'' at the University of Saarland (Saarbrücken, Germany) and LORIA (Nancy, France). 
%TODO more info
%TODO can I cite project proposal?

The ultimate goal of the project is to take initial steps toward the development of a CAPT system targeting, on the one hand, native (L1) French speakers learning German as a foreign language (L2), and on the other, L1 German speakers learning French as their L2. To this end, a bidirectional learner speech corpus has been recorded, comprising phonetically diverse utterances in French and German spoken by both native speakers and non-native speakers (i.e. speakers with the other language as L1).  %TODO cite IFCASL corpus article(s)

This thesis will focus exclusively on French L1 speakers learning German as L2. The German-language subset of the IFCASL corpus will be instrumental in training and testing the automatic diagnosis and feedback systems which this work aims to develop. Futhermore, those systems will be designed with a view to contributing to the overall set of software developed in the context of the IFCASL project, such that they will be as compatible as possible with the other tools developed and used by the IFCASL team. %TODO reword that


\section{Objectives}
\label{sec:intro:objectives}

\section{Thesis overview}
\label{sec:intro:overview}



