% Template file for an a0 portrait poster.
% Written by Graeme, 2001-03 based on his SOC poster.
%
% See discussion and documentation at
% <http://www.astro.gla.ac.uk/users/norman/docs/posters/> 
%
%%%%%%%%%%%%%%%%%%%%%%%%%%%%%%%%%%%%%%%%
% Modified by Jozef Dobo\v{s} (c) 2011 % 
%%%%%%%%%%%%%%%%%%%%%%%%%%%%%%%%%%%%%%%%

\documentclass[a0,portrait]{a0poster}
% You might find the 'draft' option to a0 poster useful if you have
% lots of graphics, because they can take some time to process and
% display. (\documentclass[a0,draft]{a0poster})

\pagestyle{empty}
\setcounter{secnumdepth}{0}
\usepackage[absolute,
					%showboxes
					]{textpos}
\usepackage{txfonts}
\usepackage{wrapfig,times}
\usepackage[scaled]{helvet}
\usepackage{graphicx}
\usepackage{forloop}
\usepackage[margin=0cm]{geometry}
\usepackage{wasysym}
\usepackage{enumitem}
\usepackage{tipa}
\usepackage{tikz}
\usetikzlibrary{backgrounds}


%%%%%%%%%%
% Colors %
%%%%%%%%%%
\usepackage{color}
\definecolor{TitleColor}{rgb}{1,1,1} % white
\definecolor{BannerOneColor}{rgb}{0,0,0} % pitch black
\definecolor{BannerTwoColor}{rgb}{0.93,0.08,0.31} % pinky red
\definecolor{BannerThreeColor}{rgb}{0,0.27,0.48} % dark blue
\definecolor{BannerFourColor}{rgb}{0.33,0.19,0.098} % brown
\definecolor{BannerSixColor}{rgb}{0,0.27,0.42} % dark blue
\definecolor{BannerSevenColor}{rgb}{0.62,0.77,0.86} % sky blue
\definecolor{BannerEightColor}{rgb}{0.35,0.33,0.01} % military green
\definecolor{BannerNineColor}{rgb}{0.85,0.86,0.34} % lime green
\definecolor{BannerTenColor}{rgb}{0,0.66,0.80} % strong blue
\definecolor{BannerElevenColor}{rgb}{0.46,0,0.20} % maroon
\definecolor{BannerTwelveColor}{rgb}{0.37,0.32,0.44} % dark washed violet
\definecolor{BannerThirteenColor}{rgb}{0.79,0.84,0.65} % light washed green
\definecolor{BannerFourteenColor}{rgb}{0.57,0.64,0.27} % dark washed green
\definecolor{BannerFifteenColor}{rgb}{0.92,0.91,0.88} % unusable washed 
\definecolor{BannerSixteenColor}{rgb}{0.94,0.36,0.14} % strong orange
\definecolor{BannerSeventeenColor}{rgb}{0.97,0.61,0.19} % orange
\definecolor{BannerEighteenColor}{rgb}{0.99,0.76,0.11} % mustard yellow
\definecolor{BannerNineteenColor}{rgb}{0.79,0.76,0.73} % light gray-ish
\definecolor{BannerTwentyColor}{rgb}{0.63,0.58,0.54} % dark gray-ish

\def\bannercolor{BannerSixColor}

%%%%%%%%%%%%%%%%%%%%%%%%%%%%%%%%%%%%%%%%%%%%%%%%%%%%%%
% Only change here to affect all headings            %
\newcommand{\headingcolor}{\color{BannerSixColor}} 
\newcommand{\titlecolor}{\color{TitleColor}}
%\newcommand{\banner}{\includegraphics[width=\linewidth]{banners/darkblue.pdf}}
\newcommand{\banner}{\LARGE \tikz{\path[draw=\bannercolor,fill=\bannercolor] (0,0) rectangle (\linewidth,2.25em);}}
\def\Highlight#1{{\sffamily \headingcolor #1}}
%%%%%%%%%%%%%%%%%%%%%%%%%%%%%%%%%%%%%%%%%%%%%%%%%%%%%%

% see documentation for a0poster class for the size options here
\let\Textsize\Large
\def\Head#1{\noindent\hbox to \hsize{\hfil{\LARGE \headingcolor #1}}\bigskip}
\def\LHead#1{\noindent{\sffamily \LARGE \headingcolor #1}\smallskip}
\def\Authors#1{\noindent{\sffamily \LARGE #1}\smallskip}
\def\Subhead#1{\noindent{\large \headingcolor #1}}
\def\Title#1{\noindent{\sffamily \VeryHuge \titlecolor #1}}

\setlist[itemize,1]{topsep=0pt, itemindent=1cm, labelsep=1cm, label={{\headingcolor $\bullet$}}}
\setlist[itemize,2]{topsep=0pt, itemindent=2cm, labelsep=1cm, label={{\headingcolor $\triangleright$}}}
%\setitemize{topsep=0pt}

% Set up the grid
%
% Note that [0cm,0cm] is the margin round the edge of the page --
% it is _not_ the grid size. That is always defined as 
% PAGE_WIDTH/HGRID and PAGE_HEIGHT/VGRID. In this case we use
% 25 x 25. This gives us three wide columns for text (7 grid
% spacings) and four narrow columns (1 each) at each side of these 
% text columns
%
% Note however that texblocks can be positioned fractionally as well,
% so really any convenient grid size can be used.
%

% [margin, margin]{rows}{cols}
\TPGrid[0cm,0cm]{17}{25}  % 1 - 7 - 1 - 7 - 1 Columns


% Mess with these as you like
\parindent=0pt
%\parindent=1cm
\parskip=0.5\baselineskip
\linespread{1.2}

% abbreviations
\newcommand{\ddd}{\,\mathrm{d}}

\begin{document}

% Understanding textblocks is the key to being able to do a poster in
% LaTeX. In
%
%    \begin{textblock}{width}(x,y)
%    ...
%    \end{textblock}
%
% the first argument gives the block width in units of the grid
% cells specified above in \TPGrid; the second gives the (x,y)
% position on the grid, with the y axis pointing down.

%%%%%%%%%%%%%%
% Top Banner %
%%%%%%%%%%%%%%


%\begin{tikzpicture}[remember picture, overlay]
%\begin{pgfonlayer}{background}
%  \path[draw=BannerTwelveColor,fill=BannerTwelveColor] (0,0) rectangle (\paperwidth,9cm);
%\end{pgfonlayer}
%\end{tikzpicture}

 %if you change this part, you can get matching color for headings
 %in Colors section above
\begin{textblock}{17}(0,0) {
%\includegraphics[width=\paperwidth]{banners/darkblue.pdf}
%\includegraphics[width=\paperwidth]{banners/purple.pdf}
\VeryHuge
\tikz{\path[draw=\bannercolor,fill=\bannercolor] (0,0) rectangle (\paperwidth,2.75em);}
} \end{textblock}




%%%%%%%%%
% Title %
%%%%%%%%%
%TODO MAKE TITLE BANNER BIGGER/LESS CRAMPED
\begin{textblock}{10}(1,0.25)
{\color{TitleColor}
\Title{A CAPT tool for training and research on lexical stress errors in German}\\

%\Authors{Anjana Vakil}, {\sffamily \Large Computational Linguistics}\\
%\textbf{\large \texttt{anjanav@coli.uni-saarland.de}}
}
\end{textblock}

\begin{textblock}{4}(12,0)    
\begin{flushright}
%\resizebox{1.5\TPHorizModule}{!}{
\includegraphics[width=1.3\TPHorizModule]{../../../img/new-owl-white.png}
%\includegraphics{images/saarland_university_white.png}
%\includegraphics{images/uds-logo-text-white.png}
\end{flushright}
\end{textblock}

\begin{textblock}{4}(10.5,0.2){
\color{TitleColor}
\begin{flushright}
\Authors{Anjana Vakil}\\
{\sffamily \Large Computational Linguistics}\\%, Saarland University}\\
{\sffamily \Large Saarland University}\\
\textbf{\large \texttt{anjanav@coli.uni-saarland.de}}
\end{flushright}
}
\end{textblock}

%\begin{textblock}{5}(1,2){
%\headingcolor
%\Authors{Anjana Vakil}\\
%{\sffamily \Large Computational Linguistics, Saarland University}\\%, U. Saarland}\\
%\textbf{\large \texttt{anjanav@coli.uni-saarland.de}}
%}
%\end{textblock}




% An example text block, to get you started!
\begin{textblock}{7}(1, 2.25) {
\banner
} \end{textblock}
\begin{textblock}{6.75}(1.25,2.5){
  \LHead{\titlecolor Introduction}}
 \end{textblock}
 \begin{textblock}{7}(1, 3.25)
  \Textsize
This poster presents the prototype Computer-Assisted Pronunciation Training (CAPT) tool \Highlight{de-stress}: the German (\Highlight{de}) \Highlight{S}ystem for \Highlight{T}raining and \Highlight{R}esearch on \Highlight{E}rrors in \Highlight{S}econd-language \Highlight{S}tress [1].

\Highlight{de-stress} targets lexical stress errors by non-native (L2) German speakers with French as their native language (L1). Its modular design incorporates various methods for diagnosing and presenting feedback on these errors, as described below. 



%\Highlight{Context:} M.Sc. thesis, related to ongoing Franco-German project\\ \textit{Individualized Feedback for Computer-Assisted Spoken Language\\ Learning} (IFCASL) [3].
%% at University of Saarland (Saarbr{\"u}cken, Germany) and LORIA (Nancy, France).


	\begin{center}
	\includegraphics[width=.7\textwidth]{../../../img/hourglass-ITS}
	\end{center}
 
\end{textblock}

\begin{textblock}{7}(1, 14.25) {
\banner
} \end{textblock}
\begin{textblock}{6.75}(1.25,14.5){
  \LHead{\titlecolor A training tool for German learners}}
 \end{textblock}
 \begin{textblock}{7}(1, 15.25)
\Textsize


	\begin{center}
	\includegraphics[width=.7\textwidth]{../../../img/screenshots/StudentInterface-userIcon}
	\end{center}

\end{textblock}


\begin{textblock}{7}(9, 2.25) {
\banner
} \end{textblock}
\begin{textblock}{6.75}(9.25,2.5){
  \LHead{\titlecolor A tool for teachers and researchers}}
 \end{textblock}
 \begin{textblock}{7}(9, 3.25)
  \Textsize
  
  An administrative interface allows language teachers or CAPT researchers to combine the available diagnostic and feedback method(s) to create new exercises for learners. 
  
  \begin{itemize}
  \item{Teachers can create exercises targeted to students' needs (learning style, etc.)}
  \item{Researchers can study impact of various diagnosis/feedback configurations on factors impacting CAPT system success, e.g.
  \begin{itemize}
  \item{learning outcomes}
  \item{user engagement/satisfaction}
  \end{itemize}
  }
  \end{itemize}
 
	\begin{center}
	\fcolorbox{gray!50}{white}{\includegraphics[width=.7\textwidth]{../FeedbackMethod}}
	\end{center}	  
  
  
\end{textblock}


%\begin{textblock}{7}(9, 8.5) {
%\banner
%} \end{textblock}
%\begin{textblock}{6.75}(9.25,8.75){
%  \LHead{\titlecolor A teaching and research tool}}
% \end{textblock}
% \begin{textblock}{7}(9, 9.5)
%  \Textsize
% 
%\end{textblock}




% Another text block in the bottom right.
\begin{textblock}{7}(9,13){
\banner
} \end{textblock}
\begin{textblock}{6.75}(9.25,13.25){
  \LHead{\titlecolor Conclusion and future directions}
   } \end{textblock}
 \begin{textblock}{7}(9, 14)
  \Textsize
  
  
\end{textblock}



\begin{textblock}{7}(9,22.5){
\banner
} \end{textblock}
\begin{textblock}{6.75}(9.25,22.75){
  \LHead{\titlecolor References}
   } \end{textblock}
 \begin{textblock}{7}(9, 23.5)
%  \Textsize
%  
%  
%\end{textblock}
%
%
%\begin{textblock}{15}(1,23)
%{\Large
%\tikz{\path[draw=\bannercolor,fill=\bannercolor] (0,0) rectangle (\linewidth,2em);}}
%\end{textblock}
%\begin{textblock}{14.75}(1.15,23.15)
%\LHead{\Large \titlecolor References}
%\end{textblock}
%\begin{textblock}{15}(1,23.55)

\begin{enumerate}[label={[\arabic*]}, leftmargin=34pt, topsep=0pt]

\item{A. S. Vakil, "de-stress," \texttt{http://github.com/vakila/de-stress}.}

\item{LORIA Speech Team, "JSnoori," \texttt{http://jsnoori.loria.fr}.}

\item{A. S. Vakil and J. Trouvain, "Automatic classification of lexical stress errors for German CAPT," in \textit{SLaTE}, 2015.}

%\item{Trouvain, J., et al. 2013. ``Designing a bilingual speech corpus for French and German language learners''. \textit{Proc. Corpus et Outils en Linguistique, Langues et Parole}, Strasbourg, pp. 32-34.}
\end{enumerate}

\end{textblock}

%\begin{textblock}{15}(1,23.25)
%\LHead{\Large \headingcolor References}
%\begin{enumerate}[label={[\arabic*]}, leftmargin=34pt, topsep=0pt]
%\item{Bonneau, A. and V. Colotte. 2011. “Automatic Feedback for L2 Prosody Learning”. In \textit{Speech and Language Technologies}. Ivo Ipsic, ed. InTech.}
%
%\item{Hirschfeld, U. 1994. \textit{Untersuchungen zur phonetischen Verst{\"a}ndlichkeit Deutschlernender.} Forum Phoneticum, 57.}
%
%\item{Trouvain, J., et al. 2013. ``Designing a bilingual speech corpus for French and German language learners''. \textit{Proc. Corpus et Outils en Linguistique, Langues et Parole}, Strasbourg, pp. 32-34.}
%\end{enumerate}
%
%\end{textblock}


% If you want to add a figure do something like this:

%\begin{textblock}{3}(1,15)
%  \begin{center}
%  \resizebox{3\TPHorizModule}{!}{\includegraphics{images/group-logo.pdf}}
%\\{\bfseries Figure 5:} caption
%  \end{center}
%\end{textblock}



% Place the group logo at the bottom left - visually this balances
% well with the University logo at the top right. 
%\begin{textblock}{4}(1,23)    
%%\resizebox{1.5\TPHorizModule}{!}{
%\includegraphics{images/uds-logo-text.png}
%%}
%\end{textblock}



\end{document}

