\documentclass[a4paper]{article}

\usepackage{INTERSPEECH2015}

\usepackage{graphicx}
\usepackage{amssymb,amsmath,bm}
\usepackage{textcomp}

\def\vec#1{\ensuremath{\bm{{#1}}}}
\def\mat#1{\vec{#1}}

\usepackage{cleveref}

\usepackage[dvipsnames]{xcolor}
\newcommand{\TODO}[1]{{\color{red}\textbf{[TODO #1]}}}

\sloppy % better line breaks
\ninept

\title{The de-stress CAPT tool: A System for Training and Research on Errors in Second-language Stress in German}

%%%%%%%%%%%%%%%%%%%%%%%%%%%%%%%%%%%%%%%%%%%%%%%%%%%%%%%%%%%%%%%%%%%%%%%%%%
%% If multiple authors, uncomment and edit the lines shown below.       %%
%% Note that each line must be emphasized {\em } by itself.             %%
%% (by Stephen Martucci, author of spconf.sty).                         %%
%%%%%%%%%%%%%%%%%%%%%%%%%%%%%%%%%%%%%%%%%%%%%%%%%%%%%%%%%%%%%%%%%%%%%%%%%%
%\makeatletter
%\def\name#1{\gdef\@name{#1\\}}
%\makeatother
%\name{{\em Firstname1 Lastname1, Firstname2 Lastname2, Firstname3 Lastname3,}\\
%      {\em Firstname4 Lastname4, Firstname5 Lastname5, Firstname6 Lastname6,
%      Firstname7 Lastname7}}
%%%%%%%%%%%%%%% End of required multiple authors changes %%%%%%%%%%%%%%%%%

\makeatletter
\def\name#1{\gdef\@name{#1\\}}
\makeatother \name{{\em%
  Anjana Sofia Vakil
  %Author Name$^1$, Co-author Name$^2$
  }}

\address{%
  %$^1$Author Affiliation \\
  %$^2$Co-author Affiliation \\
  Department of Computational Linguistics \& Phonetics\\
  Saarland University, Saarbr\"{u}cken, Germany\\
  {\small \tt 
  anjanav@coli.uni-saarland.de}
}
%\twoauthors{Karen Sp\"{a}rck Jones.}{Department of Speech and Hearing \\
%  Brittania University, Ambridge, Voiceland \\
%  {\small \tt Karen@sh.brittania.edu} }
%  {Rose Tyler}{Department of Linguistics \\
%  University of Speechcity, Speechland \\
%  {\small \tt RTyler@ling.speech.edu} }

%
\begin{document}

  \maketitle
  %
  \begin{abstract}
    This demonstration presents a prototype tool for Computer-Assisted Pronunciation Training (CAPT) in German. The 
  \end{abstract}
  \noindent{\bf Index Terms}: speech recognition, human-computer interaction, computational paralinguistics


	\section{System overview}
	\label{sec:overview}
	
	\TODO{}
	
	\section{Error diagnosis options}
	\label{sec:diag}
	
	\TODO{}
	
	\section{Feedback options}
	\label{sec:fb}
	
	\TODO{}
	
	


  %\newpage
  \eightpt
  \bibliographystyle{IEEEtran}

  \bibliography{mybib}

%  \begin{thebibliography}{9}
%    \bibitem[1]{Davis80-COP}
%      S.\ B.\ Davis and P.\ Mermelstein,
%      ``Comparison of parametric representation for monosyllabic word recognition in continuously spoken sentences,''
%      \textit{IEEE Transactions on Acoustics, Speech and Signal Processing}, vol.~28, no.~4, pp.~357--366, 1980.
%    \bibitem[2]{Rabiner89-ATO}
%      L.\ R.\ Rabiner,
%      ``A tutorial on hidden Markov models and selected applications in speech recognition,''
%      \textit{Proceedings of the IEEE}, vol.~77, no.~2, pp.~257-286, 1989.
%    \bibitem[3]{Hastie09-TEO}
%      T.\ Hastie, R.\ Tibshirani, and J.\ Friedman,
%      \textit{The Elements of Statistical Learning -- Data Mining, Inference, and Prediction}.
%      New York: Springer, 2009.
%    \bibitem[4]{YourName15-XXX}
%      F.\ Lastname1, F.\ Lastname2, and F.\ Lastname3,
%      ``Title of your INTERSPEECH 2015 publication,''
%      in \textit{Interspeech 2015 -- 16\textsuperscript{th} Annual Conference of the International Speech Communication Association, September 06--10, Dresden, Germany, Proceedings}, 2015, pp.~100--104.
%  \end{thebibliography}

\end{document}
