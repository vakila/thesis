\documentclass[xcolor={dvipsnames}]{beamer}

%%% PACKAGES %%%
\usepackage{graphicx}
\usepackage{tabularx}
\usepackage{booktabs}
\usepackage{multirow}
%\usepackage[usenames,dvipsnames]{color}
\usepackage{textpos}
\usepackage{tipa}
\usepackage{hyperref}
%\usepackage{pifont}



%%% FONT %%%
\usepackage{tgheros}


%%% BEAMER THEME %%%
\usetheme{default}
\usecolortheme{whale}
\usecolortheme{orchid}
\setbeamertemplate{navigation symbols}{} 
\setbeamertemplate{footline}[frame number]
%\setbeamertemplate{footline}{%
%  \leavevmode%
%  \hbox{%
%    \pgfsetfillopacity{0}\begin{beamercolorbox}[wd=.333333\paperwidth,ht=2.25ex,dp=1ex,center]{author in head/foot}%
%      \usebeamerfont{author in head/foot}\pgfsetfillopacity{1}\insertshortauthor
%    \end{beamercolorbox}%
%    \pgfsetfillopacity{0}\begin{beamercolorbox}[wd=.333333\paperwidth,ht=2.25ex,dp=1ex,center]{title in head/foot}%
%      \usebeamerfont{title in head/foot}\pgfsetfillopacity{1}\insertshorttitle
%    \end{beamercolorbox}%
%    \pgfsetfillopacity{0}\begin{beamercolorbox}[wd=.333333\paperwidth,ht=2.25ex,dp=1ex,right]{date in head/foot}%
%      \usebeamerfont{date in head/foot}\pgfsetfillopacity{1}\insertshortdate{}\hspace*{2em}
%      \insertframenumber{} / \inserttotalframenumber\hspace*{2ex}
%    \end{beamercolorbox}}%
%  \vskip0pt%
%}
\addtobeamertemplate{frametitle}{}{%
\begin{textblock*}{100mm}(.875\textwidth,-.9cm)
\includegraphics[height=.9cm]{../../../img/uni_des_saarlandes-white.png}
\end{textblock*}}
\setbeamertemplate{caption}{\insertcaption}
%\setbeamerfont{caption}{size=\scriptsize}
\setbeamertemplate{itemize subitem}[circle]
%\setbeamertemplate{bibliography item}[text]
\setbeamertemplate{bibliography item}[triangle]
\setbeamertemplate{blocks}[rounded][shadow=true]


%%% COLORS %%%
\definecolor{MutedBlue}{RGB}{83,121,170}
\definecolor{UniBlue}{RGB}{0,71,116}
\definecolor{DarkPink}{RGB}{178,56,119}
\definecolor{LightPink}{RGB}{255,224,246}
\definecolor{LightBlue}{RGB}{214,255,255}

%\setbeamercolor{title}{fg=UniBlue}
%\setbeamercolor{frametitle}{fg=UniBlue}
\setbeamercolor{author in head/foot}{fg=UniBlue}
\setbeamercolor{title in head/foot}{fg=UniBlue}
\setbeamercolor{date in head/foot}{fg=UniBlue}
\setbeamercolor{page number in head/foot}{fg=UniBlue}
\setbeamercolor{normal text}{fg=darkgray}
\setbeamercolor{structure}{fg=UniBlue}
\setbeamercolor{itemize item}{fg=UniBlue}
\setbeamercolor{itemize subitem}{fg=UniBlue}
%\setbeamercolor{section in toc}{fg=MutedBlue}
%\setbeamercolor{subsection in toc}{fg=darkgray}
%\setbeamercolor{bibliography entry author}{fg=darkgray}
%\setbeamercolor{bibliography entry title}{fg=darkgray}
%\setbeamercolor{bibliography entry location}{fg=gray!80!black}
%\setbeamercolor{bibliography entry note}{fg=gray!90!black}
%\setbeamercolor{bibliography entry note}{fg=darkgray}
\setbeamercolor{block title}{bg=structure,fg=white}
%\setbeamercolor{block body}{bg={palette secondary}}


%%% FOOTNOTES %%%
\usepackage{perpage}
\MakePerPage{footnote} %restart numbering on each slide
\renewcommand{\footnotesize}{\scriptsize}

%%% BIBLIOGRAPHY %%%
\usepackage[%
	backend=bibtex,
	citestyle=authoryear,
	maxcitenames=2,
	maxbibnames=99,
	firstinits=true,
	url=false,
	doi=false,
	isbn=false, 
	%sorting=none,
	]{biblatex}
\addbibresource{../slaterefs.bib}
% use \footfullcite{jones00} or \footcite{jones00}

%% TITLE PAGE INFO %%
\author[A. Vakil \& J. Trouvain]{Anjana Vakil and J\"{u}rgen Trouvain
%\\\texttt{[anjanav,trouvain]@coli.uni-saarland.de}
} 
\institute{\includegraphics[height=4.5em]{../../../img/uni_des_saarlandes.png}\\Department of Computational Linguistics and Phonetics\\Saarland University, Saarbr{\"u}cken, Germany}
\title{Automatic classification of lexical stress errors\\for German CAPT}
%\subtitle{Investigating the impact of source language choice} 
\date[4.9.15]{SLaTE 2015, Leipzig\\4 September 2015}

%% CUSTOM COMMANDS %%
\newcommand{\TODO}[1]{{\color{red}\textbf{[TODO #1]}}}

%% DOCUMENT %%
\begin{document}
{
\setbeamertemplate{footline}{} 
\begin{frame}
  \titlepage
\end{frame}
}
\addtocounter{framenumber}{-1}

%\begin{frame}
%\frametitle{Outline \TODO{skip this slide?}}
%\tableofcontents%[pausesections]
%\end{frame}


\section{Background \& motivation}
\begin{frame}{Lexical stress \TODO{(LS)} in German}
%\begin{definition}
%\structure{Lexical stress:} 
%Prosodic 
Accentuation/prominence of syllable(s) in a word%\footnotemark
%\end{definition}
\vfill
%\addtocounter{footnote}{-1}
In German:
\begin{itemize}
\item{Variable placement, contrastive function%\footcite{Cutler2005}%\footnotemark
%\footcite{Cutler2005}
\begin{center}
\begin{tabular}{ccc}
um$\cdot$FAHR$\cdot$en & vs. & UM$\cdot$fahr$\cdot$en \\
\textit{to drive around} & & \textit{to run over}\\
\end{tabular}
\end{center}
}
\item{Reflected by duration, fundamental frequency (F0), intensity\footcite{Dogil1999}}
\item{Impacts intelligibility of non-native (L2) speech\footcite{Hirschfeld1994}}
%\item{Contrastive LS notoriously difficult for French speakers\footcite{Dupoux1997}}
\end{itemize}
\vfill
\end{frame}

%\begin{frame}{Lexical stress in German}
%\begin{itemize}
%\item{Impacts intelligibility of non-native (L2) German speech\footcite{Hirschfeld1994}}
%\item{Challenge for learners with other native languages (L1s),\\e.g. French\footcite{Michaux2013}}
%\item{Huge potential for individualized instruction via Computer-Assisted Pronunciation Training (CAPT)}
%\item{Reliable automatic detection of German lexical stress errors (LSEs) not yet established}
%\end{itemize}
%\vfill
%%
%%Goal: Computer-Assisted Pronunciation Training (CAPT) for lexical stress errors for French learners of German
%\end{frame}

\begin{frame}{CAPT for lexical stress errors \TODO{(LSEs)}}
\begin{itemize}
\item{Contrastive LS notoriously difficult for French speakers%\footcite{Michaux2013}%
\footcite{Dupoux1997}
}
\item{CAPT offers huge potential for individualized instruction}
\vfill
%\pause
%\item{Most CAPT for prosody uses comparison-based approach}
%\item{Typical approach: comparison with native (L1) reference speaker}
%\item{Risk: too dependent on speaker or utterance}
%\vfill
%\pause
\item{Classification of LS errors in L2 German unexplored}
\item{Promising recent work using machine learning for classification of English stress patterns\footcite{Kim2011,Shahin2012a}
%\begin{itemize}
%\item{50-60\% accuracy on low-quality L2 English speech}
%\end{itemize}
}
\end{itemize}
\vfill
%\pause
\structure{Our goal}: explore classification-based detection of lexical stress errors by French learners of German
\end{frame}


\section{Data}
\subsection{The IFCASL Franco-German corpus}

\begin{frame}{Data}
Subset of IFCASL corpus of French-German speech\footcite{Fauth2014}
		\begin{figure}
		\TODO{new screenshot}
		\includegraphics[width=\textwidth]{../../../colloquium/2SH05_FGMB1_527-flagge}
		\end{figure}

%	\begin{itemize}
%			\item{German sentences read by L1/L2 speakers}
%			\item{Automatically segmented at phone, syllable, word level}
%			\item{No lexical stress error annotation}
%	\end{itemize}
	
%\vfill
%\vspace{1em}

%Manual annotation of lexical stress errors
	Extracted utterances of 12 bisyllabic, initial-stress words
	\begin{itemize}
		%\item{Extracted utterances of 12 bisyllabic, initial-stress words}
		\item{668 tokens from 56 French speakers - manually annotated}
		\item{477 tokens from 40 German speakers - assumed correct}
	\end{itemize}
\end{frame}

\begin{frame}{Data annotation}
\begin{itemize}
\item{Each token assigned a class label:

	\begin{center}
	\begin{tabular}{ll}
	3 stress classes: & 2 error classes: \\
	$[$correct$]$, $[$incorrect$]$, $[$none$]$ &  $[$bad\_nsylls$]$, $[$bad\_audio$]$ \\
	\end{tabular}
	\end{center}
	
}
\item{15 annotators (12 native), each token labeled by $\geq$2}
%\item{Pairwise inter-annotator agreement varies; low overall}
\end{itemize}

		\begin{table}
			\centering
			\small
			\caption{Overall pairwise inter-annotator agreement}
			%{\color{blue}
			\begin{tabularx}{\textwidth}{lXXXX}
			\toprule
			 								& Mean & Maximum & Median & Minimum \\
			 \midrule
			 \% Agreement 		& 54.92\% & 83.93\% & 55.36\% & 23.21\% \\
			 Cohen's $\kappa$	& 0.23 & 0.61 & 0.26 & -0.01 \\
			 \bottomrule
			\end{tabularx}					
			\label{tab:agreement:overall}
		\end{table}

\begin{itemize}
\item{Variability not explained by L1 or expertise}
\item{Single gold-standard label selected for each token}
\end{itemize}

\end{frame}


%%%%%%%%% OLD Data slides %%%%%%%%%%

%\begin{frame}{Data \TODO{OLD - remove}}
%IFCASL corpus of French-German speech\footcite{Fauth2014}
%		\begin{itemize}
%			\item{Phonetically diverse German sentences}
%			\item{Read aloud by L1 and L2 (L1 French) speakers %French and German speakers
%				\begin{itemize}
%				\item Adults ($>$18) and children (15-16)
%				\item Levels\footnote{Common European Framework of Reference, www.coe.int/lang-CEFR} A2, B1, B2, C1 (children all A2/B1)
%				\end{itemize}
%			}
%			\item{Automatic (forced-alignment) segmentation at phone, syllable, and word level}
%			\item{No lexical stress error annotation}
%		\end{itemize}
%		
%\end{frame}
%
%\subsection{Annotation of lexical stress errors}
%\begin{frame}{Data annotation \TODO{figure?} \TODO{OLD - remove}}
%%	\begin{figure}
%%	\includegraphics[width=\textwidth]{../../../colloquium/2SH05_FGMB1_527-flagge}
%%	\end{figure}
%%	\vfill
%	
%	Manual annotation of LSEs in subset of corpus:\\
%	\begin{itemize}
%	\item{12 bisyllabic, initial-stress words (word types)
%%		\begin{itemize}
%%		\item{Bisyllabic}
%%		\item{Stress on initial syllable}
%%		\end{itemize}
%	}
%	\item{Word utterances (tokens) extracted automatically}
%	\item{%\hspace*{20pt} 
%	668 tokens from $\sim$55 French speakers}
%	%TODO include this? \item{\TODO{X} tokens from L1 German speakers assumed to be error-free; not manually annotated}
%	\end{itemize}
%
%	\vfill
%	Each token assigned a class label:
%	\begin{center}
%	\begin{tabular}{ll}
%	3 stress classes: & 2 error classes: \\
%	$[$correct$]$ &  $[$bad\_nsylls$]$ \\
%	$[$incorrect$]$ &  $[$bad\_audio$]$ \\
%	$[$none$]$ & \\
%	\end{tabular}
%	\end{center}
%	
%	
%\end{frame}
%
%\begin{frame}{Data annotation \TODO{OLD - remove}}
%	%\begin{itemize}
%%	\item 15 annotators, varying by:
%%		\begin{itemize}
%%		\item L1 (12 German, 2 English, 1 Hebrew)
%%		\item Phonetics/phonology expertise \\(2 expert, 10 intermediate, 3 novice)
%%		\end{itemize}
%	%\item 
%	15 annotators (12 native), each token labeled by $\geq$2 
%	%\item 
%	
%	\vfill
%	
%	Pairwise agreement calculated using:
%		\begin{itemize}
%		\item Percentage agreement: $N$ agreed $/$ $N$ both annotated
%		\item Cohen's Kappa %\footcite{Cohen1960} 
%		($\kappa$): accounts for chance agreement
%		\end{itemize}
%	%\end{itemize}
%\vfill
%%Pairwise agreement varies considerably, rather low overall
%
%		\begin{table}
%			\centering
%			\small
%			%\caption{Overall pairwise agreement}
%			%{\color{blue}
%			\begin{tabularx}{\textwidth}{lXXXX}
%			\toprule
%			 								& Mean & Maximum & Median & Minimum \\
%			 \midrule
%			 \% Agreement 		& 54.92\% & 83.93\% & 55.36\% & 23.21\% \\
%			 Cohen's $\kappa$	& 0.23 & 0.61 & 0.26 & -0.01 \\
%			 \bottomrule
%			\end{tabularx}					
%			\label{tab:agreement:overall}
%		\end{table}
%	
%	\vfill
%	Single gold-standard label selected for each token
%	%Variability not explained by L1 or expertise	
%
%\end{frame}


\begin{frame}{Data annotation results}

	\begin{figure}
		\centering
		%\caption{Overall distribution of lexical stress errors in annotated data}
		\includegraphics[width=\textwidth]{../overallJudgments-withLabels}
		\label{fig:results:overallbars}
	\end{figure}
\end{frame}

\section{Method}

\begin{frame}{Method}
%		Experiments:
%		\begin{itemize}
%		\item Train CART classifiers using WEKA toolkit\footnote{\texttt{www.cs.waikato.ac.nz/ml/weka}}
%		\item Use error-annotated L2 utterances for training/testing\\(gold-standard labels)
%		\item Use L1 utterances of same words for training (all [correct])
%		\end{itemize}
		
		Train \& evaluate CART classifiers using WEKA toolkit\footnote{\texttt{www.cs.waikato.ac.nz/ml/weka}}
		
		\vfill
		
		\structure{Training data}
		\begin{itemize}
			\item Manually annotated L2 utterances
			\item Automatically annotated L1 utterances (all [correct])
		\end{itemize}				
		
		\vfill
		
		\structure{Held-out testing data}
		\begin{itemize}
			\item{Feature comparison: 1/10 of L2 utterances (random)}
			\item{Unseen speakers: all utterances from 1 of 56 L2 speakers}
		\end{itemize}
		
		\vfill
		
		\structure{Evaluation}
		\begin{itemize}
			\item Compute agreement (\% and $\kappa$) with gold standard
			\item Average across 10 or 56 folds
		\end{itemize}
		
%		Evaluated in terms of:
%		\begin{itemize}
%		\item \% accuracy (\% agreement with gold-standard labels) 
%		\item $\kappa$ with respect to gold standard
%		\end{itemize}
		
		\end{frame}
		
	\subsection{Feature sets}
%		\begin{frame}{Feature sets}
%		\textit{Which features are most useful for classification?}
%		\vfill
%		\begin{tabularx}{\textwidth}{lX}
%			%\begin{tabular}{ll}
%			\toprule
%			Feature set & Description \\
%			\midrule
%			DUR & Duration features \\
%			F0 & Fundamental frequency features \\
%			INT & Intensity features \\
%			\midrule
%			WD %WORD 
%				& Uttered word (e.g. \textit{Tatort}) \\
%			LV % LEVEL 
%				& Speaker's skill level  (A2$|$B1$|$B2$|$C1)\\
%			AG %GENDER 
%				& Speaker's age/gender  (Girl$|$Boy$|$Woman$|$Man)\\
%%			LV+AG & LEVEL, GENDER \\
%%			WD+LV & WORD, LEVEL \\
%%			WD+AG & WORD, GENDER \\
%%			WD+SPKR & WORD, LEVEL, GENDER \\
%			\bottomrule
%			\end{tabularx}		
%		\end{frame}

		\begin{frame}{Feature sets}
		
%		Best predictors of German stress: duration, fundamental frequency (F0), intensity\footcite{Dogil1999}
%		
%		\vspace{1em}

		\structure{Prosodic feature sets}
			\begin{itemize}
			\item{DUR - Duration (relative syllable \& nucleus lengths)
%				\begin{itemize}
%					%\item{Best indicator of German stress\footcite{Dogil1999}}
%					\item{Features: relative syllable \& nucleus length}
%				\end{itemize}
			}
			\item{F0 - Fundamental frequency (mean, max., min., range)
%				\begin{itemize}
%					%\addtocounter{footnote}{-1}
%					%\item Next best indicator after duration\footnotemark
%					\item Pitch contours computed using JSnoori\footcite{DiMartino1999}
%					\item{Features: relative F0 mean, max., min. and range in syllable \& nucleus}
%				\end{itemize}
			}
			\item{INT - Intensity (mean, max.)
%				\begin{itemize}
%					%\item Worse indicator than duration/F0, but may still affect stress perception\footcite{Cutler2005}
%					\item 
%					\item Features: relative mean and maximum energy of syllable \& nucleus
%				\end{itemize}
			}
			\end{itemize}		
			
%			\vspace{1em}
%			
%			Listed in order of importance for German stress\footcite{Dogil1999}
			
			\vspace{1em}
			
			Pitch and energy contours calculated using JSnoori software\footnote{\texttt{jsnoori.loria.fr}}%$^{,}$\footcite{DiMartino1999}
			
			\vfill
			
			For German stress, duration seemingly best indicator, then F0\footcite{Dogil1999}
		
		\end{frame}
		
		\begin{frame}{Feature sets}
%		\textit{Which features are most useful for classification?}
%		\vfill
%		\begin{itemize}
		\structure{Prosodic feature sets}
			\begin{itemize}
			\item DUR - Duration
			\item F0 - Fundamental frequency
			\item INT - Intensity 
			\end{itemize}
			
		\vspace{1em}	
			
		\structure{Other features}
			\begin{itemize}
			\item WD - Word uttered (e.g. \textit{Flagge})
%			\item Speaker features
%			\begin{itemize}
			\item LV - Speaker's skill level (A2$|$B1$|$B2$|$C1)
			\item AG - Speaker's age/gender (Girl$|$Boy$|$Woman$|$Man)
			\end{itemize}
%		\end{itemize}

		%\vfill
		\end{frame}


%\subsection{Training/testing datasets}
%\begin{frame}{Training/testing datasets}
%\end{frame}

\section{Results}
	\begin{frame}{Results}
		%\textit{How well can lexical stress errors be classified?}
		\begin{figure}
		\includegraphics<1>[width=\textwidth]{../../../colloquium/results-prosodicfeatures-newaxes}
		%\includegraphics<2>[width=\textwidth]{../../../colloquium/results-prosodicfeatures-newaxes-highlight}
		\includegraphics<2->[width=\textwidth]{results-prosodicfeatures-newaxes-highlight-label-black}
		\end{figure}
		
		%\vfill
		%\pause
		
%		\begin{figure}
%		\includegraphics<1>[width=\textwidth]{../../../colloquium/results-speakerword-durf0-neweraxes}
%		\includegraphics<2>[w80idth=\textwidth]{../../../colloquium/results-speakerword-all-neweraxes}
%		\includegraphics<3>[width=\textwidth]{../../../colloquium/results-speakerword-all-neweraxes-highlight}
%		\end{figure}
		
%		%\vfill
%		\pause
%		Best performance using only prosodic features: DUR+F0
%		\begin{itemize}
%		\item \% Accuracy: 69.77\% 
%		\item $\kappa$: 0.29
%		\end{itemize}
	\end{frame}
		
	\begin{frame}{Results}
		%\textit{How well can lexical stress errors be classified?}
		\begin{figure}
		\includegraphics<1>[width=\textwidth]{../../../colloquium/results-speakerword-durf0-neweraxes}
		\includegraphics<2>[width=\textwidth]{../../../colloquium/results-speakerword-all-neweraxes}
		%\includegraphics<3>[width=\textwidth]{../../../colloquium/results-speakerword-all-neweraxes-highlight}
		\includegraphics<3>[width=\textwidth]{results-speakerword-all-neweraxes-highlight-label}
		\end{figure}
		
%		\pause
%		\pause
%		Best performance overall: WD+LV+DUR+F0+INT
%		\begin{itemize}
%		\item \% Accuracy: 71.87\%
%		\item $\kappa$: 0.34
%		\end{itemize}
	\end{frame}
	
	
	\begin{frame}{Results}
		\begin{figure}
		\includegraphics<1>[width=\textwidth]{results-unseenspeakers}
		\includegraphics<2->[width=\textwidth]{results-unseenspeakers-highlight-label}
		\end{figure}
	\end{frame}
	
	\begin{frame}{Results}
		\textit{How does classification accuracy compare \\with human agreement?}
		\vfill
		\begin{tabularx}{\textwidth}{Xcc}
		\toprule
		& \% agreement & $\kappa$ \\
		\midrule
		Best classifier vs. gold standard & 71.87\% & 0.34\\
		Mean human vs. human & 54.92\%	&	0.23\\
		\bottomrule
		\end{tabularx}
		\vfill
		\begin{itemize}
		\item Results are encouraging in this context
		\item Still want better performance for real-world use
		\end{itemize}
	\end{frame}
	


\section{Conclusions \& future work}
\begin{frame}{Conclusions}
\end{frame}

\section{References}
\begin{frame}[noframenumbering]{Selected references}
\renewcommand*{\bibfont}{\footnotesize}
    \printbibliography

%{\scriptsize
%	\begin{itemize}
%	\item \fullcite{Cutler2005}
%	\item \fullcite{Michaux2013}
%	\item \fullcite{Hirschfeld1994}
%	\item \fullcite{Bonneau2011}
%	\item \fullcite{Kim2011}
%	\item \fullcite{Fauth2014}
%	%\item \fullcite{Mesbahi2011}
%	\item \fullcite{Dogil1999}
%	\item \fullcite{DiMartino1999}
%	\item \fullcite{Probst2002}
%	\item \fullcite{Neri2002}
%	\end{itemize}
%}
\end{frame}
%}

\end{document}